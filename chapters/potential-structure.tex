%% -*- coding:utf-8 -*-
\chapter{Structure, potential structure and underspecification}
\label{chap-potenital-structure}

The previous chapter extensively dealt with the question whether one should adopt a phrasal or a
lexical analysis of valence alternations. This rather brief chapter deals with a related issue. I
discuss the analysis of complex predicates consisting of a preverb and a light
verb. Preverbs often have an argument structure of their own. They describe an event and the light
verb can be used to realize either the full number of arguments or a reduced set of
arguments. (\mex{1}) provides the example from Hindi\il{Hindi}\il{Urdu} discussed by \citet*{VRP2019a}.


\eal
\ex
\gll logon=ne      pustak=kii tareef k-ii\\
      people.\textsc{m}=\textsc{erg} book.\textsc{f}=\textsc{gen} praise.\textsc{f} do-\textsc{perf}.\textsc{f}\\
\glt `People praised the book.' Lit: `People did praise of the book.'
\ex
\gll pustak=kii tareef hu-ii\\
     book.\textsc{f}.\textsc{sg}=\textsc{gen} praise.\textsc{f} be.\textsc{part}-\textsc{perf}.\textsc{f}.\textsc{sg}/be.\textsc{pres}\\
\glt `The book got praised.' Lit: `The praise of the book happened.'
\zl
\emph{tareef} `praise' is a noun that can be combined with the light verb \emph{kar} `do' to form an
active sentence as in (\mex{0}a) or with the light verb \emph{ho} `be' to form a passive sentence as in (\mex{0}b).
Similar examples can of course be found in other languages making heavy use of complex predicates
\citep{MuellerPersian}.

In what follows I compare the analysis of \citet{VRP2019a} in the framework of Lexicalized TAG with
a an HPSG analysis. As the name specifies, LTAG
is a lexicalized framework, something that was argued for in the previous chapter. However, TAG is
similar to phrasal Construction Grammar in that it makes use of phrasal configurations to represent
argument slots. This differs from Categorial Grammar\indexcg and HPSG\indexhpsg since the latter frameworks assume
descriptions of arguments (head/functor representations in CG and valence lists in HPSG) rather than
structures containing these arguments. So while TAG\indextag elementary trees contain actual structure, CG
and HPSG lexical items contain potential structure. TAG structures can be taken appart and items can
be inserted into the middle of an existing structure but usually the structure is not transformed
into something else.\footnote{%
  One way to ``delete'' parts of the structure would be to assume empty elements that can be
  inserted into substitution nodes (see Chapter~\ref{chap-empty} for discussion).
}
This is an interesting difference that becomes crucial when talking about the formation of complex predicates and in particular about certain active/passive alternations. 

\citet{VRP2019a} assume that the structures for the examples in (\mex{0}) are composed of elementary
trees for \emph{tareef} `praise' and the respective light verbs. This is shown in Figure~\ref{fig-hindi-lv-active} and
Figure~\ref{fig-hindi-lv-passive} respectively.
%\addlines
\largerpage
\begin{figure}
\hfill%
\begin{forest}
tag
[XP$_0$
  [NP$_1$  $\downarrow$]
  [\fbox{XP$_1$} 
    [NP$_2$ $\downarrow$]
    [XP$_2$
      [X      [tareef;praise]]]]]
\end{forest}
\hfill%
\begin{forest}
tag
[XP$_r$
    [XP$_f$] 
    [VP
      [V      [k-ii;do]]]]
\end{forest}
\hfill%
\begin{forest}
tag
[XP$_0$
  [NP$_1$ [logon{=}ne;people]]
  [XP$_1$ [XP$_f$
            [NP$_2$ [pustak{=}kii;book]]
            [XP$_2$ [X      [tareef;praise]]]]
    [VP [V [k-ii;do]]]]]
\end{forest}
\hfill\mbox{}
\caption{Analysis of \emph{logon=ne      pustak=kii tareef k-ii} `People praised the book.' The tree
of the light verb is adjoined into the tree of the preverb, into the XP$_1$ position}\label{fig-hindi-lv-active}
\end{figure}
\begin{figure}
\hfill
\begin{forest}
tag
[\fbox{XP$_1$}
  [NP$_1$  $\downarrow$]
  [XP$_2$
    [X      [tareef;praise]]]]
\end{forest}
\hfill%
\begin{forest}
tag
[XP$_r$
    [XP$_f$] 
    [VP
      [V      [hu-ii;be]]]]
\end{forest}
\hfill%
\begin{forest}
tag
[XP$_1$ [XP$_f$
            [NP$_1$ [pustak{=}kii;book]]
            [XP$_2$ [X      [tareef;praise]]]]
    [VP [V [hu-ii;be]]]]
\end{forest}
\hfill\mbox{}
\caption{Analysis of \emph{pustak=kii tareef hu-ii} `The book got praised.'}\label{fig-hindi-lv-passive}
\end{figure}
%\addlines
\largerpage
The TAG analysis is only sketched here. The authors use feature-based TAG\indexftag, which makes it possible
to enforce obligatory adjunction: the elementary tree for \emph{tareef} is specified in a way that
makes it necessary to take the tree apart and insert nodes of another tree (see page~\pageref{page-feature-based-tag-oa}). This way it can be
ensured that the preverb has to be augmented by a light verb. This results in XP$_f$ being inserted
at XP$_1$\todostefan{check} in the figures above.

What the analysis clearly shows is that TAG assumes two lexical items for the preverb: one with two
arguments for the active case and one with just one argument
for the passive. In general one would
say that \emph{tareef} is a noun describing the praising event, that is, one 
person praises another one. Now this noun can be combined with a light verb and depending on which
light verb is used we get an active sentence with both arguments realized or a passive sentence with
the agent of the eventive noun suppressed. There is no morphological reflex of this active/passive
alternation at the noun. It is just the same noun \emph{tareef}: in an active sentence in (\mex{0}a)
and in a passive one in (\mex{0}b).

%\largerpage
And here we see a real difference between the frameworks: TAG is a framework in which structure is
assembled: the basic operations are substitution and adjunction. The lexicon consists of ready-made
building blocks that are combined to yield the trees we want to have in the end. This differs from
Categorial Grammar %\citep{Ajdukiewicz35a-u} 
and HPSG %\citep{ps2,Sag97a} 
where lexical items do not
encode real structure to be used in an analysis, but potential structure: lexical items come with a
list of their arguments, that is, items that are required for the lexical element under
consideration to project to a full phrase. However, lexical heads may enter
relations with their valents and form NPs, APs, VPs, PPs or other phrases, but they do not have
to. \citet{Geach70a} developed a technique that is called functional composition\is{function composition} or argument
composition\is{argument composition}\is{argument attraction} within the framework of Categorial Grammar and this was transferred to HPSG by
\citet{HN89a,HN94a}. Since the 90ies this technique is used for the analysis of complex predicates
in HPSG for German \citep{HN89a,HN94a,Kiss95a,Meurers99a,Mueller99a,Kathol2000a},
Romance\il{Romance} (\citealp[\page 600]{MS97a-u}; \citealp{Monachesi98a}; \citealp{AG2002b-u}),
Korean \citep{Chung98a-u}, and Persian\il{Persian} \citep{MuellerPersian}. See \citew{GS2020a} for
an overview. For instance \citet[\page
  642]{MuellerPersian} analyzes the light verbs \emph{kardan} `do' and \emph{šodan} `become' this
way: both raise the subject of the embedded predicate and make it their own argument but
\emph{kardan} introduces an additional argument while \emph{šodan} does not do so.

Applying the argument composition technique to our example, we get the following lexical item for
\emph{tareef}:
\ea
Sketch of lexical item for \emph{tareef} `praise':\\
\ms{
head   & \ms[noun]{
         subj & \sliste{ \ibox{1} }\\
         }\\
comps  & \sliste{ \ibox{2} NP }\\
arg-st & \sliste{ \ibox{1} NP, \ibox{2} NP }\\
}
\z
\largerpage
The \argstl contains all arguments of a head. The arguments are linked to the semantic representation
and are mapped to valence features like \textsc{specifier} and \textsc{complements}. Depending on
the langauge and the realizationability of subjects within projections, that subject may be mapped to
a separate feature, which is a \headf. \headfs are projected along the head path but the features
contained under \head do not license combinations with the head.

The lexical items for \emph{kar} `do' and  \emph{ho} `be' are:
\eal
\ex Sketch of lexical item for \emph{kar} `do':\\
\ms{
head & verb\\
arg-st & \ibox{1} $\oplus$ \ibox{2} $\oplus$ \sliste{ N[\subj \ibox{1}, \comps \ibox{2} ] } 
}
\ex Sketch of lexical item for \emph{ho} `be':\\
\ms{
head & verb\\
arg-st & \ibox{1} $\oplus$ \sliste{ N[\comps \ibox{1} ] } 
}
\zl
The verb \emph{kar} `do' selects for a noun and takes whatever the subject of this noun is \iboxb{1} and 
concatenates the list of complements the noun takes \iboxb{2} with the value of \subj. The result is \ibox{1} $\oplus$
\ibox{2} and it is a prefix of the \argstl of the light verb. 
The lexical item for \emph{ho} `be' is similar, the difference being that the subject of the
embedded verb is not attracted to the higher \argstl, only the complements \iboxb{1} are.

For finite verbs it is assumed that all arguments are mapped to the \compsl of the verb, so the
\compsl is identical to the \argstl.
\largerpage[-1]
The analysis of our example sentences is shown in Figures~\ref{fig-hindi-lv-active-hpsg} and~\ref{fig-hindi-lv-passive-hpsg}.
\begin{figure}[t]
\hfill%
\begin{forest}
sm edges
[V
   [\ibox{1} NP [{logon=ne};{people.\textsc{m}{=}\textsc{erg}}]]
   [V
     [\ibox{2} NP [{pustak=kii};{book.\textsc{f}{=}\textsc{gen}}]]
     [V
        [\ibox{3} N\feattab{
            \subj  \sliste{ \ibox{1} }\\
            \comps  \sliste{ \ibox{2} }\\
            \argst  \sliste{ \ibox{1}, \ibox{2} }} [tareef;{praise.\textsc{f}}]]
        [V\feattab{
            \subj  \sliste{ }\\
            \comps  \ibox{4}\\
            \argst  \ibox{4} \sliste{ \ibox{1}, \ibox{2}, \ibox{3} }} [k-ii;{do-\textsc{perf}.\textsc{f}}]]]]]
\end{forest}
\hfill\mbox{}
\caption{Analysis of \emph{logon=ne      pustak=kii tareef k-ii} `People praised the book.' The
  arguments of the preverb are taken over by the light verb}\label{fig-hindi-lv-active-hpsg}
\end{figure}


\begin{figure}[t]
\hfill%
\begin{forest}
sm edges
   [V
     [\ibox{2} NP [{pustak=kii};{book.\textsc{f}{=}\textsc{gen}}]]
     [V
        [\ibox{3} N\feattab{
            \subj  \sliste{ \ibox{1} NP }\\
            \comps  \sliste{ \ibox{2} }\\
            \argst  \sliste{ \ibox{1}, \ibox{2} }} [tareef;{praise.\textsc{f}}]]
        [V\feattab{
            \subj  \sliste{ }\\
            \comps  \ibox{4}\\
            \argst  \ibox{4} \sliste{ \ibox{2}, \ibox{3} }} [hu-ii;{be.\textsc{part}-\textsc{perf}.\textsc{f}.\textsc{sg}/be.\textsc{pres}}]]]]]
\end{forest}
\hfill\mbox{}
\caption{Analysis of \emph{pustak=kii tareef hu-ii} `The book got praised.'}\label{fig-hindi-lv-passive-hpsg}
\end{figure}

The conclusion is that HPSG has a representation of potential structure. When light verbs are present,
they can take over valents and ``execute'' them according to their own preferences. This is not
possible in TAG since once structure is assembled it cannot be changed. We may insert items into
the middle of an already assembled structure but we cannot take out arguments or reorder them. This
is possible in Categorial Grammar and in HPSG: the governing head may choose which arguments to take
order and in which order they should be represented in the valence repsresentations of the governing head.

LFG is somewhere in the middle between TAG and HPSG: the phrase structural configurations are not
fully determined as in TAG since LFG does not store and manipulate phrase markers. But lexical items are
associated with f-structures and these f-structures are responsible for which elements are realized
in syntax. As complex predicates are assumed to be monoclausal it is not sufficient to embed the
f-structure of the preverb within the f-structure of the light verb \citep{BHKM2003a-u}. Since the grammatical functions
that are ultimately realized in the clause do not depend on the preverb alone the light verb may have
to determine the grammatical functions contributed by the preverb. In order to be able to do this \citet{BHKM2003a-u}
use the restriction operator \citep{KW93a-u}, which restricts out certain features or path equations provided by the
preverb's and the light verb's f-structures. The statement of grammatical functions in
f-structures is another instance of too strict specifications: once specified, it is difficult to get rid of it and special means like partial copying via
restriction are needed. 
An alternative not relying on restriction was suggested by \citet{Butt97a}:
embedding relations can be specified on the a-structure representation and then a mapping is defined
that maps the complex a-structure to the desired f-structure. Mapping between several levels of
representation is a general tool that is also used in HPSG: for instance, \citet*{BMS2001a} used
\argst, \deps, and \comps in the treatment of nonlocal dependencies. See also \citet{Koenig99a} on
the introduction of arguments via additional auxiliary features. As I showed in
\citet[Section~7.5.2.2]{MuellerLehrbuch1}, one would need an extra feature for every kind of argument alternation
that is to be modeled this way. Recent versions of LFG use glue semantics to keep track of arguments
(\citealp*{DLS93a-u}; \citealp[Chapter~8]{Dalrymple2001a-u}). Glue semantics\is{glue semantics} can
be used to do argument extension and argument manipulation in general in ways that are parallel to
argument attraction approaches. See for instance \citew*{ADT2013a} for a treatment of
benefactive\is{benefactive} arguments. 


%\largerpage
Summing up, I showed that there are indeed differences between the frameworks that are due to the
basic representational formalisms they assume. While TAG assumes that the lexicon contains
trees with a certain structure, HPSG assumes that lexical items come with valence specifications,
that is, they have descriptions of items that have to be combined with the lexical item. But the way
in which the items have to be combined with the head is determined by dominance schemata (grammar
rules) that are separate from the lexical items. So the valence specifications specify possible
structures. Since valence
representations can be composed by superordinate predicates there is enough flexibility to deal with
various light verb phenomena. LFG is a bit more constrained due to the use of f-structures, but using
a restriction operator unwanted information about grammatical functions can be keept out of
f-structures of matrix predicates.

\bigskip
\furtherreading{

This chapter is based on \citet{MuellerPotentialStructure}.
}


%      <!-- Local IspellDict: en_US-w_accents -->






