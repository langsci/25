%% -*- coding:utf-8 -*-

\section*{Foreword of the third edition}

% fixed \forwardt for Harry in 7-cg.tex Ina Baier 18.06.2018

% added mention of Rizzi2014a 09.07.2018

% fixed 0,045 which should have been 0.045 in innateness chapter.

% added reference/source for she smiled herself an upgrade.

% added reference to Chesi2015a because of infinite sentences

% changed the discussion of passive in German in GB a bit to make things clearer, 10.10.2018

% added references to GSag2000a and NK2019a in the discussion of start symbol and utterance
% fragments 18.01.2019

% fixed Baumgärtner's dicussion of rule schema rather than rule 09.02.2019

% fixed missing italics in Figure for Max sleeps.

Since more and more researchers and students are using the book now, I get feedback that helps
improving it. For the third edition I added references, expanded the discussion of the passive in GB
a bit and fixed typos.

Chapter~\ref{chap-mp} contained figures from different chapters of \citew{Adger2003a}. Adger
introduces the DP rather late in the book and I had a mix of NPs and DPs in figures. I fixed this in
the new edition. I am so used to talk about NPs that there were references to NP in the general
discussion that should have been references to DP. I fixed this as well.

I thank Frank Van Eynde for pointing out eight typos in his review of the first edition. They have
been fixed. He also pointed out that the placement of \argst in the feature geometry of signs in
HPSG did not correspond to \citew{GSag200a}, where \argst is on the top-level rather than under
\cat. Note that earlier versions of this book had \argst under \cat and it was never properly argued
to not have it there which is why many practitioners of HPSG kept it at this place
\citep{MuellerLFGphrasal}. One reason to keep \argst on the top level is that \argst is appropriate
for lexemes only. If \argst is on the sign level this can be represented in the type hierarchy:
lexemes and word have an \argst feature, phrases do not. If \argst is on the \cat level one would
have to distinguish between \catvs that belong to lexemes and words on the one hand and phrasal
\catvs on the other hand, which would require two additional subtypes of the type \type{cat}. 
The most recent version of the computer implementation done in Stanford by Dan Flickinger has \argst
under \local (2019-01-24). So, I was tempted to leave everything as it was in the second edition of
the book. However, there is a real argument for not having \argst under \cat. \cat is assumed to be
shared in coordinations. \cat contains valence features for subjects and complements. The values of
these valence features are determined by a mapping from \argst. In some analyses extracted elements
are not mapped to the valence features and the same is sometimes assumed for omitted elements. To
take an example consider (\mex{1}):
\ea
He saw and helped the hikers.
\z
\emph{saw} and \emph{helped} are coordinated and the members in the valence lists have to be
compatible. Now if one coordinates a ditransitive verb with one omitted argument with a strictly
transitive verb, this would work under the assumption that the omitted argument is not part of the
valence representation. But if \argst is part of \cat, coordination would be made impossible since a
three-place argument structure list would be incompatible with a two-place list. Hence I decided to
change this in the third edition and represent \argst outside of \cat from now on.


I thank Mikhail Knyazev for pointing out to me that the treatment of V to I to C movement in the
German literature differs from the lowering that is assumed for English and that some further
references are needed in the chapter on Government \& Binding. 

I thank Hubert Haider, Gisbert
Fanselow and Pavel Logacev for discussion and Ina Baier for a mistake in a CG proof and Jonas Benn for pointing out some typos to me.

~\medskip

\noindent
Berlin, \today\hfill Stefan Müller



%      <!-- Local IspellDict: en_US-w_accents -->
