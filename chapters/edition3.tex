%% -*- coding:utf-8 -*-

\section*{Foreword of the third edition}

% fixed \forwardt for Harry in 7-cg.tex Ina Baier 18.06.2018

% added mention of Rizzi2014a 09.07.2018

% fixed 0,045 which should have been 0.045 in innateness chapter.

% added reference/source for she smiled herself an upgrade.

% added reference to Chesi2015a because of infinite sentences

% changed the discussion of passive in German in GB a bit to make things clearer, 10.10.2018

% added references to GSag2000a and NK2019a in the discussion of start symbol and utterance
% fragments 18.01.2019

% fixed Baumgärtner's dicussion of rule schema rather than rule 09.02.2019

% fixed missing italics in Figure for Max sleeps.

% fixed ungrammatical Chinese example in phrasal.tex 30.07.2019, Thanks Wang Lulu 

Since more and more researchers and students are using the book now, I get feedback that helps
improve it. For the third edition I added references, expanded the discussion of the passive in GB (Section~\ref{sec-passive-gb})
a bit and fixed typos.\footnote{%
  A detailed list of issues and fixes can be found in the GitHub repository of this book at
  \url{https://github.com/langsci/25/}.%
}

Chapter~\ref{chap-mp} contained figures from different chapters of \citew{Adger2003a}. Adger
introduces the DP rather late in the book and I had a mix of NPs and DPs in figures. I fixed this in
the new edition. I am so used to talking about NPs that there were references to NP in the general
discussion that should have been references to DP. I fixed this as well. I added a figure explaining
the architecture in the Phase model of Minimalism and since the figures mention the concept of
\emph{numeration}, I added a footnote on numerations. I also added a figure depicting the
architecture assumed in Minimalist theories with Phases (right figure in Figure~\ref{fig-architecture-minimalism}).

I thank Frank Van Eynde for pointing out eight typos in his review of the first edition. They have
been fixed. He also pointed out that the placement of \argst in the feature geometry of signs in
HPSG did not correspond to \citew{GSag2000a-u}, where \argst is on the top level rather than under
\cat. Note that earlier versions of this book had \argst under \cat and there had never been proper
arguments for why it should not be there, which is why many practitioners of HPSG have kept it in
that position \citep{MuellerLFGphrasal}. One reason to keep \argst on the top level is that \argst is appropriate
for lexemes only. If \argst is on the sign level, this can be represented in the type hierarchy:
lexemes and words have an \argst feature, phrases do not. If \argst is on the \cat level, one would
have to distinguish between \catvs that belong to lexemes and words on the one hand and phrasal
\catvs on the other hand, which would require two additional subtypes of the type \type{cat}. 
The most recent version of the computer implementation done in Stanford by Dan Flickinger has \argst
under \local (2019-01-24). So, I was tempted to leave everything as it was in the second edition of
the book. However, there is a real argument for not having \argst under \cat. \cat is assumed to be
shared in coordinations and \cat contains valence features for subjects and complements. The values of
these valence features are determined by a mapping from \argst. In some analyses, extracted elements
are not mapped to the valence features and the same is sometimes assumed for omitted elements. To
take an example consider (\mex{1}):
\ea
He saw and helped the hikers.
\z
\emph{saw} and \emph{helped} are coordinated and the members in the valence lists have to be
compatible. Now if one coordinates a ditransitive verb with one omitted argument with a strictly
transitive verb, this would work under the assumption that the omitted argument is not part of the
valence representation. But if \argst is part of \cat, coordination would be made impossible since a
three-place argument structure list would be incompatible with a two-place list. Hence I decided to
change this in the third edition and represent \argst outside of \cat from now on.\footnote{
  Note added on 2021-11-05: The editors of the HPSG handbook \citep*{HPSGHandbook} decided to put
  \argst under \cat \citep[\page 19]{Abeille:Borsley2021a} because of the analysis of \isi{complex predicates} in \ili{French}. On French complex
  predicates see \citew[\page 426--427]{GS2021a}.
}

I changed the section about Sign-Based Construction Grammar (SBCG) again. An argument about nonlocal
dependencies and locality was not correct, since \citet[\page 166]{Sag2012a} does not share all
information between filler and extraction side. The argument is now revised and presented as
Section~\ref{sec-local-feature-sbcg}. Reviewing \citew{MuellerCxG}, Bob Borsley pointed out to me that the \xargf is a way to
circumvent locality restrictions that is actually used in SBCG. I added a footnote to the section on
locality in SBCG.

A brief discussion of \citegen{Welke2019a-u} analysis of the German clause structure was added to the
chapter about Construction Grammar (see Section~\ref{sec-verb-position-cxg}).

The analysis of a verb-second sentence in LFG is now part of the LFG chapter
(Figure~\ref{Abbildung-V2-LFG} on page~\pageref{Abbildung-V2-LFG}) and not just an
exercise in the appendix. A new exercise was designed instead of the old one and the old one was
integrated into the main text.

I added a brief discussion of \citegen{Osborne2019a} claim that Dependency Grammars are simpler than
phrase structure grammars (p.\,\pageref{page-simplicity-dg}).

Geoffrey Pullum pointed out at the HPSG conference in 2019 that the label \emph{constraint"=based}
may not be the best for the theories that are usually referred to with it. Changing the term in
this work would require to change the title of the book. The label \emph{model theoretic} may be
more appropriate but some implementational work in HPSG and LFG not considering models may find the
term inappropriate. I hence decided to stick to the established term.

I followed the advice by Lisbeth Augustinus and added a preface to Part~II of the book that gives
the reader some orientation as to what to expect.

I thank Mikhail Knyazev for pointing out to me that the treatment of V to I to C movement in the
German literature differs from the lowering that is assumed for English and that some further
references are needed in the chapter on Government \& Binding. 

Working on the Chinese translation of this book, Wang Lulu pointed out some
typos and a wrong example sentence in Chinese. Thanks for these comments! 

I thank Bob Borsley, Gisbert Fanselow, Hubert Haider and Pavel Logacev for discussion and Ina Baier for a mistake
in a CG proof and Jonas Benn for pointing out some typos to me. Thanks to Tabea Reiner for a comment
on gradedness. Thanks also to Antonio Machicao y Priemer for yet another set of comments on the
second edition and to Elizabeth Pankratz for proofreading parts of what I changed.

~\medskip

\noindent
Berlin, 15th August 2019\hfill Stefan Müller



%      <!-- Local IspellDict: en_US-w_accents -->
