\chapter{Preface}

This book is an extended and revised version of my German book \emph{Grammatiktheorie}
\citep{MuellerGTBuch2}. It introduces various grammatical theories that play a role in current
theorizing or have made contributions in the past which are still relevant today. I explain some foundational
assumptions and then apply the respective theories to what can be called the ``core grammar'' of
German. I have decided to stick to the object language that I used in the German version of this
book since many of the phenomena that will be dealt with cannot be explained with English as the object
language. Furthermore, many theories have been developed by researchers with English as their native
language and it is illuminative to see these theories applied to another language.
I show how the theories under consideration deal with arguments and adjuncts, active/passive
alternations, local reorderings (so"=called scrambling), verb position, and fronting of phrases over
larger distances (the verb second property of the Germanic languages without English).

The second part deals with foundational questions that are important for developing theories.
This includes a discussion of the question of whether we have innate domain specific knowledge of
language (UG), the discussion of psycholinguistic evidence concerning the processing of language by
humans, a discussion of the status of empty elements and of the question whether we construct and perceive utterances 
holistically or rather compositionally, that is, whether we use phrasal or lexical
constructions. The second part is not intended as a standalone book although the printed version of
the book is distributed this way for technical reasons (see below). Rather it contains topics that are discussed again and again when frameworks are
compared. So instead of attaching these discussions to the individual chapters they are organized in
a separate part of the book.

Unfortunately, linguistics is a scientific field 
with a considerable amount of terminological chaos. I therefore wrote an introductory
chapter that introduces terminology in the way it is used later on in the book. The second chapter
introduces phrase structure grammars, which plays a role for many of the theories that are covered
in this book. I use these two chapters (excluding the Section~\ref{sec-PSG-Semantik} on interleaving
phrase structure grammars and semantics) in introductory courses of our BA curriculum for German
studies. Advanced readers may skip these introductory chapters. The following chapters are
structured in a way that should make it possible to understand the introduction of the theories
without any prior knowledge. The sections\label{page:structure-of-book} regarding new developments and classification are more
ambitious: they refer to chapters still to come and also point to other publications that are
relevant in the current theoretical discussion but cannot be repeated or summarized in this
book. These parts of the book address advanced students and researchers. I use this book for teaching
the syntactic aspects of the theories in a seminar for advanced students in our BA. The slides are
available on my web page. The second part of the book, the general discussion, is more ambitious and contains the discussion
of advanced topics and current research literature.

This book only deals with relatively recent developments. For a historical overview, see for instance
\citew{Robins97a-u,JL2006a-u}. I am aware of the fact that chapters on
Integrational Linguistics\is{Integrational Linguistics}
\citep{Lieb83a-u,Eisenberg2004a,Nolda2007a-u}, Optimality Theory\indexot (\citealp{PS93a-u};
\citealp{Grimshaw97a-u}; G.\ \citealp{GMueller2000a-u}), Role and Reference Grammar\is{Role and
  Reference Grammar} \citep{vanValin93a-ed} and Relational Grammar\is{Relational Grammar}
\citep{Perlmutter83a-ed,PR84a-ed} are missing. I will leave these theories for later editions.

The original German book was planned to have 400 pages, but it finally was much bigger: the first
German edition has 525 pages and the second German edition has 564 pages. I
added a chapter on Dependency Grammar and one on Minimalism to the English version and now the
book has \pageref{LastPage} pages. I tried to represent the chosen theories appropriately and to cite all important work. Although the list of
references is over 85 pages long, I was probably not successful.
I apologize for this and any other shortcomings.

%%% -*- coding:utf-8 -*-

\section*{Available versions of this book}

The canonical version of this book is the PDF document available from the Language Science Press
webpage of this book\footnote{%
\url{\lsURL}
}. This page also links to a Print on Demand version. Since the book is very long, we decided to
split the book into two volumes. The first volume contains the description of all theories and the
second volume contains the general discussion. Both volumes contain the complete list of references
and the indices. The second volume starts with page~\pageref{part-discussion}. The printed volumes
are therefore identical to the parts of the PDF document.







%      <!-- Local IspellDict: en_US-w_accents -->


\section*{Acknowledgments}


I would like to thank David Adger\aimention{David Adger},
Jason Baldridge\aimention{Jason Baldridge}, 
Felix Bildhauer\aimention{Felix Bildhauer}, 
Emily M.\ Bender\aimention{Emily M. Bender},
Stefan Evert\aimention{Stefan Evert}, 
Gisbert Fanselow\aimention{Gisbert Fanselow}, 
Sandiway Fong\aimention{Sandiway Fong},
Hans"=Martin Gärtner\aimention{Hans"=Martin G"artner},
Kim Gerdes\aimention{Kim Gerdes},
Adele Goldberg\aimention{Adele Goldberg},
Bob Levine\aimention{Robert D. Levine},
Paul Kay\aimention{Paul Kay},
Jakob Maché\aimention{Jakob Mach{\'e}},
Guido Mensching\aimention{Guido Mensching},
Laura Michaelis\aimention{Laura Michaelis},
Geoffrey Pullum\aimention{Geoffrey K. Pullum}, 
Uli Sauerland\aimention{Uli Sauerland}, 
Roland Schäfer\aimention{Roland Sch"afer},
Jan Strunk\aimention{Jan Strunk},
Remi van Trijp\aimention{Remi van Trijp}, 
Shravan Vasishth\aimention{Shravan Vasishth},
Tom Wasow\aimention{Tom Wasow}, and
Stephen Wechsler\aimention{Stephen Mark Wechsler}
for discussion and 
%
Monika Budde\aimention{Monika Budde}, 
Philippa Cook\aimention{Philippa Helen Cook},
Laura Kallmeyer\aimention{Laura Kallmeyer}, 
Tibor Kiss\aimention{Tibor Kiss},
Gisela Klann"=Delius\aimention{Gisela Klann"=Delius}, 
Jonas Kuhn\aimention{Jonas Kuhn},
Timm Lichte\aimention{Timm Lichte}, % für Kommentare zum TAG"=Kapitel
Anke Lüdeling\aimention{Anke L"udeling},
Jens Michaelis\aimention{Jens Michaelis},
Bjarne Ørsnes\aimention{Bjarne {\O}rsnes},
Andreas Pankau\aimention{Andreas Pankau},     % Chomsky 2013
Christian Pietsch\aimention{Christian Pietsch},
Frank Richter\aimention{Frank Richter},
Ivan Sag\aimention{Ivan A. Sag}, and
Eva Wittenberg\aimention{Eva Wittenberg}
for comments on earlier versions of the German edition of this book and
%
%
Thomas Groß\aimention{Thomas M. Gro{\ss}},
Dick Hudson\aimention{Richard Hudson},
Sylvain Kahane\aimention{Sylvain Kahane}, 
Paul Kay\aimention{Paul Kay},
Haitao Liu (刘 海涛)\aimention{Haitao Liu},
Andrew McIntyre\aimention{Andrew McIntyre},
Sebastian Nordhoff\aimention{Sebastian Nordhoff},
Tim Osborne\aimention{Timothy Osborne}, 
Andreas Pankau\aimention{Andreas Pankau}, and
Christoph Schwarze\aimention{Christoph Schwarze}
for comments on earlier versions of this book. Thanks to Leonardo Boiko and Sven Verdoolaege for pointing out typos.
Special thanks go to Martin Haspelmath\aimention{Martin Haspelmath} for very detailed comments on an
earlier version of the English book. 

This book was the first Language Science Press book that had an open review phase (see below). I
thank Dick Hudson, Paul Kay, Antonio Machicao y Priemer, Andrew McIntyre, Sebastian Nordhoff, and one anonymous open
reviewer for their comments. Theses comments are documented at the \href{\lsURL}{download page of
  this book}. In addition the book went through a stage of community proofreading (see also
below). Some of the proofreaders did much more than proofreading, their comments are highly
appreciated and I decided to publish these comments as additional open reviews.
Armin Buch, 
Leonel de Alencar,
Andreas Hölzl,
Gianina Iordăchioaia,
Timm Lichte,
Antonio Machicao y Priemer, and
Neal Whitman
deserve special mention here.


I thank Wolfgang Sternefeld and Frank Richter, who wrote a detailed review of the German version of
this book \citep{SR2012a}. They pointed out some mistakes and omissions that were corrected in the second edition
of the German book and which are of course not present in the English version.

Thanks to all the students who commented on the book and whose questions lead to improvements. 
Lisa Deringer,
Aleksandra Gabryszak, % Student SS 2010 gute Fragen, GB-Verbbewegung und LMT
Simon Lohmiller, %Student, Typos und allgemeine Anregung zu Einführungskapitel
Theresa Kallenbach, %Studentin, GPSG
Steffen Neu\-schulz,  % Student SS 2010 gute Fragen
Reka Meszaros-Segner,
Lena Terhart and
Elodie Winckel deserve special mention.

Since this book is built upon all my experience in the area of grammatical theory, I want to thank
all those with whom I ever discussed linguistics during and after talks at conferences, workshops,
summer schools or via email.
Werner Abraham,
John Bateman,
Dorothee Beermann, 
Rens Bod, 
Miriam Butt,
Manfred Bierwisch,
Ann Copestake, 
Holger Diessel, 
Kerstin Fischer,
Dan Flickinger,
Peter Gallmann, 
%Adele Goldberg,  included above
Petter Haugereid,
Lars Hellan, 
% Paul Kay, included above
Tibor Kiss,
Wolfgang Klein, 
Hans"=Ulrich Krieger,
%Emily M. Bender, included above
Andrew McIntyre,
Detmar Meurers,
%Laura Michaelis,  included above
Gereon Müller,
Martin Neef,
Manfred Sailer, 
Anatol Stefanowitsch,
Peter Svenonius,
Michael Tomasello, 
Hans Uszkoreit, 
Gert Webelhuth, 
% Stephen Wechsler included above
Daniel Wiechmann and Arne Zeschel deserve special mention.

I thank Sebastian Nordhoff for a comment regarding the completion of the subject index entry for \emph{recursion}\is{recursion}.

Andrew Murphy translated part of Chapter~1 and the Chapters~2--3, 5--10, and 12--23. Many thanks for this!

I also want to thank the 27 community proofreaders (\makeatletter\@proofreader\makeatother) that each worked on one or more chapters and
really improved this book. I got more comments from every one of them than I ever got for a book
done with a commercial publisher. Some comments were on content rather than on typos and layout
issues. No proofreader employed by a commercial publisher would have spotted these mistakes and
inconsistencies since commercial publishers do not have staff that knows all the grammatical
theories that are covered in this book. 

During the past years, a number of workshops on theory comparison have taken place. I was invited to three of them.
I thank Helge Dyvik\aimention{Helge Dyvik} and Torbjørn Nordgård\aimention{Torbj{\o}rn
  Nordg{\r{a}}rd}\todostefan{Indexeinträge für Torbjorn and Bjarne do not work} for inviting me to the fall school for Norwegian PhD
students  \emph{Languages and Theories in Contrast}, which took place 2005 in Bergen. Guido Mensching\aimention{Guido Mensching} and Elisabeth
Stark\aimention{Elisabeth Stark} invited me to the workshop \emph{Comparing Languages and Comparing Theories:
  Generative Grammar and Construction Grammar}, which took place in 2007 at the Freie Universität
Berlin and Andreas Pankau\aimention{Andreas Pankau} invited me to the workshop \emph{Comparing
  Frameworks} in 2009 in Utrecht. I really enjoyed the discussion with all participants of these
events and this book benefited enormously from the interchange.

I thank Peter Gallmann\aimention{Peter Gallmann} for the discussion of his lecture notes on \gb
during my time in Jena. The Sections~\ref{Abschnitt-T-Modell}--\ref{Abschnitt-GB-Passiv} have a
structure that is similar to the one of his script and take over a lot. Thanks to David Reitter for
the \LaTeX{} macros for Combinatorial Categorial Grammar, to Mary Dalrymple and Jonas Kuhn for the LFG
macros and example structures, and to Laura Kallmeyer for the \LaTeX{} sources of most of the TAG
analyses. Most of the trees have been adapted to the \texttt{forest} package because of compatibility issues
with \XeLaTeX, but the original trees and texts were a great source of inspiration and without them
the figures in the respective chapters would not be half as pretty as they are now.

I thank Sašo Živanović for implementing the \LaTeX{} package \texttt{forest}. It really simplifies
typesetting of trees, dependency graphs, and type hierarchies. I also thank him for individual help
via email and on \href{http://www.stackexchange.com}{stackexchange}. In general, those active on stackexchange could not be thanked
enough: most of my questions regarding specific details of the typesetting of this book or the
implementation of the \LaTeX{} classes that are used by \lsp now have been answered within several
minutes. Thank you! Since this book is a true open access book under the CC-BY license, it can also
be an open source book. The interested reader finds a copy of the source code at \url{https://github.com/langsci/25}. By making the book open source I pass on the knowledge provided by the \LaTeX{} gurus and
hope that others benefit from this and learn to typeset their linguistics papers in nicer and/or
more efficient ways.


Viola Auermann and Antje Bahlke, Sarah Dietzfelbinger, Lea Helmers, and Chiara Jancke cannot be thanked enough for their work at the copy machines. Viola
also helped a lot with proof reading prefinal stages of the translation.
I also want to thank my (former) lab members Felix Bildhauer, Philippa Cook, Janna Lipenkova, Jakob Maché,
Bjarne Ørsnes and Roland Schäfer\aimention{Roland Sch{\"a}fer}, which were mentioned above already
for other reasons, for their help with teaching. During the years from 2007 until the publication of
the first German edition of this book two of the three tenured positions in German Linguistics were
unfilled and I would have not been able to maintain the teaching requirements without their help and
would have never finished the \emph{Grammatiktheorie} book.

I thank Tibor Kiss for advice in questions of style. His diplomatic way always was a shining
example for me and I hope that this is also reflected in this book.

%      <!-- Local IspellDict: en_US-w_accents -->

\section*{On the way this book is published}

\addlines[2]
I started to work on my dissertation in 1994 and defended it in 1997. During the whole time the
manuscript was available on my web page. After the defense, I had to look for a publisher. I was
quite happy to be accepted to the series \emph{Linguistische Arbeiten} by Niemeyer, but at the same time I
was shocked about the price, which was 186.00 DM for a paperback book that was written and typeset
by me without any help by the publisher (twenty times the price of a paperback novel).\footnote{%
  As a side remark: in the meantime Niemeyer was bought by de Gruyter and closed down. The price of the book is now
  139.95 \euro{} / \$ 196.00. The price in Euro corresponds to 273.72 DM. Update 23.06.2020: The book
  is sold for 149.95 \euro{} / \$ 169,82 now.
%This is a price increase of 47\,\%.
} This
basically meant that my book was depublished: until 1998 it was available from my web page and after
this it was available in libraries only. My Habilitationsschrift was published by CSLI Publications
for a much more reasonable price. When I started writing textbooks, I was looking for alternative
distribution channels and started to negotiate with no-name print on demand publishers. Brigitte Narr,
who runs the Stauffenburg publishing house, convinced me to publish my HPSG textbook with her. The
copyrights for the German version of the book remained with me so that I could publish it on my web page. The collaboration was successful so that I also published my second textbook about
grammatical theory with Stauffenburg. I think that this book has a broader relevance and should be
accessible for non-German-speaking readers as well. I therefore decided to have it translated into
English. Since Stauffenburg is focused on books in German, I had to look for another publisher. Fortunately the situation in the publishing sector changed quite dramatically in comparison
to 1997: we now have high profile publishers with strict peer review that are entirely open access. I am very
glad about the fact that Brigitte Narr sold the rights of my book back to me and that I can now 
publish the English version with Language Science Press under a CC-BY license.



%      <!-- Local IspellDict: en_US-w_accents -->

\section*{Language Science Press: scholar-owned high quality linguistic books}

In 2012 a group of people found the situation in the publishing business so unbearable that they
agreed that it would be worthwhile to start a bigger initiative for publishing linguistics books in
platinum open access, that is, free for both readers and authors. I set up a web page and collected
supporters, very prominent linguists from all over the world and all subdisciplines and Martin
Haspelmath and I then founded Language Science Press. At about the same time the DFG had announced
a program for open access monographs and we applied \citep{MH2013a} and got funded (two out of 18 applications got
funding). The money is used for a coordinator (Dr.\ Sebastian Nordhoff) and an economist (Debora
Siller), two programmers (Carola Fanselow and Dr.\ Mathias Schenner), who work on the publishing
plattform Open Monograph Press (OMP) and on conversion software that produces various formats (ePub, XML,
HTML) from our \LaTeX{} code. Svantje Lilienthal works on the documentation of OMP, produces
screencasts and does user support for authors, readers and series editors.

OMP is extended by open review facilities and community-building gamification tools
\citep{MuellerOA,MH2013a}. All Language Science Press books are reviewed by at least two external
reviewers. Reviewers and authors may agree to publish these reviews and thereby make the whole
process more transparent (see also \citew{Pullum84a} for the suggestion of open reviewing of journal
articles). In addition there is an optional second review phase: the open
review. This review is completely open to everybody. The whole community may comment on the document
that is published by Language Science Press. After this second review phase, which usually lasts for
two months, authors may revise their publication and an improved version will be published. This
book was the first book to go through this open review phase. The annotated open review version of this book is still available via
the \href{\lsURL}{web page of this book}. 

Currently, Language Science Press has 17 series on various subfields of linguistics with high
profile series editors from all continents. We have 18 published and 17 forthcoming books and 146
expressions of interest. Series editors  and authors are responsible for
delivering manuscripts that are typeset in \LaTeX{}, but they are supported by a web-based typesetting
infrastructure that was set up by Language Science Press and by volunteer typesetters from the
community. Proofreading is also community-based. Until now 53 people helped improving our
books. Their work is documented in the Hall of Fame: \url{http://langsci-press.org/about/hallOfFame}.

If you think that textbooks like this one should be freely available to whoever wants to read them
and that publishing scientific results should not be left to profit-oriented publishers, then you
can join the Language Science Press community and support us in various ways: you can register with Language Science Press and have your name
listed on our supporter page with almost 600 other enthusiasts, you may devote your time and help
with proofreading and/or typesetting, or you may donate money for specific books or for Language
Science Press in general. We are also looking for institutional supporters like foundations,
societies, linguistics departments or university libraries. Detailed information on how to support
us is provided at the following webpage: \url{http://langsci-press.org/about/support}.
In case of questions, please contact me or the Language Science Press coordinator at \href{mailto:contact@langsci-press.org}{contact@langsci-press.org}.


~\medskip

\noindent
Berlin, \today\hfill Stefan Müller


%      <!-- Local IspellDict: en_US-w_accents -->


%% -*- coding:utf-8 -*-

\section*{Forword of the second edition}

I want to thank Wang Lulu for pointing out some typos that she found while translating the book to
Chinese. Thanks for both the typos and the translation.

Fritz Hamm noticed that the definition of Intervention was incomplete and pointed out some
inconsistencies in translations of predicates in Section~\ref{sec-PSG-Semantik}.

% In GB-Kapitel what = roof
%
I turned some straight lines in Chapter~\ref{chap-GB} into triangles and added a discussion of
different ways to represent movement.

%Hamm: add Heim/Kratzer, Quantifier-Movement erklären

I extended the discussion of Pirahã in Section~\ref{sec-recursion-empirical-problems} and added
lexical items that show that Pirahã-like modification without recursion can be captured in a
straightforward way in Categorial Grammar. 

Sašo Živanović helped adapting version 2.0 of the \texttt{forest} package so that it could be used
with this large book. I am very graceful for this nice tree typesetting package and all the work
that went into it.

% SpecIP Begriff erklärt. Fußnote zu Spec als Label in Bäumen.

% Added Riemsdijk78:148 as first reference against the toolbox approach to UG.

% Bresnan94a zitiert: Dutch (Maling & Zaenen 1978, Perlmutter & Zaenen 1984), Icelandic and Faroese (Platzack 1987)
%Similarly, \citet[Section~4]{Safir85a-u} assumes that impersonal passives\is{passive!impersonal} in

% Philippa two typos

% Timm Lichte: Transformation kommt bei Chomsky so nicht vor: Übersetzungsfehler

% 18.06.2017
% removed [4] and [5] in schemata with MOTHER feature and translation to PSG
%
% updated URLs

~\medskip

\noindent
Berlin, \today\hfill Stefan Müller



%      <!-- Local IspellDict: en_US-w_accents -->

%% -*- coding:utf-8 -*-

\section*{Foreword of the third edition}

% fixed \forwardt for Harry in 7-cg.tex Ina Baier 18.06.2018

% added mention of Rizzi2014a 09.07.2018

% fixed 0,045 which should have been 0.045 in innateness chapter.

% added reference/source for she smiled herself an upgrade.

% added reference to Chesi2015a because of infinite sentences

% changed the discussion of passive in German in GB a bit to make things clearer, 10.10.2018

% added references to GSag2000a and NK2019a in the discussion of start symbol and utterance
% fragments 18.01.2019

% fixed Baumgärtner's dicussion of rule schema rather than rule 09.02.2019

% fixed missing italics in Figure for Max sleeps.

% fixed ungrammatical Chinese example in phrasal.tex 30.07.2019, Thanks Wang Lulu 

Since more and more researchers and students are using the book now, I get feedback that helps
improve it. For the third edition I added references, expanded the discussion of the passive in GB (Section~\ref{sec-passive-gb})
a bit and fixed typos.\footnote{%
  A detailed list of issues and fixes can be found in the GitHub repository of this book at
  \url{https://github.com/langsci/25/}.%
}

Chapter~\ref{chap-mp} contained figures from different chapters of \citew{Adger2003a}. Adger
introduces the DP rather late in the book and I had a mix of NPs and DPs in figures. I fixed this in
the new edition. I am so used to talking about NPs that there were references to NP in the general
discussion that should have been references to DP. I fixed this as well. I added a figure explaining
the architecture in the Phase model of Minimalism and since the figures mention the concept of
\emph{numeration}, I added a footnote on numerations. I also added a figure depicting the
architecture assumed in Minimalist theories with Phases (right figure in Figure~\ref{fig-architecture-minimalism}).

I thank Frank Van Eynde for pointing out eight typos in his review of the first edition. They have
been fixed. He also pointed out that the placement of \argst in the feature geometry of signs in
HPSG did not correspond to \citew{GSag2000a-u}, where \argst is on the top level rather than under
\cat. Note that earlier versions of this book had \argst under \cat and there had never been proper
arguments for why it should not be there, which is why many practitioners of HPSG have kept it in
that position \citep{MuellerLFGphrasal}. One reason to keep \argst on the top level is that \argst is appropriate
for lexemes only. If \argst is on the sign level, this can be represented in the type hierarchy:
lexemes and words have an \argst feature, phrases do not. If \argst is on the \cat level, one would
have to distinguish between \catvs that belong to lexemes and words on the one hand and phrasal
\catvs on the other hand, which would require two additional subtypes of the type \type{cat}. 
The most recent version of the computer implementation done in Stanford by Dan Flickinger has \argst
under \local (2019-01-24). So, I was tempted to leave everything as it was in the second edition of
the book. However, there is a real argument for not having \argst under \cat. \cat is assumed to be
shared in coordinations and \cat contains valence features for subjects and complements. The values of
these valence features are determined by a mapping from \argst. In some analyses, extracted elements
are not mapped to the valence features and the same is sometimes assumed for omitted elements. To
take an example consider (\mex{1}):
\ea
He saw and helped the hikers.
\z
\emph{saw} and \emph{helped} are coordinated and the members in the valence lists have to be
compatible. Now if one coordinates a ditransitive verb with one omitted argument with a strictly
transitive verb, this would work under the assumption that the omitted argument is not part of the
valence representation. But if \argst is part of \cat, coordination would be made impossible since a
three-place argument structure list would be incompatible with a two-place list. Hence I decided to
change this in the third edition and represent \argst outside of \cat from now on.\footnote{
  Note added on 2021-11-05: The editors of the HPSG handbook \citep*{HPSGHandbook} decided to put
  \argst under \cat \citep[\page 19]{Abeille:Borsley2021a} because of the analysis of \isi{complex predicates} in \ili{French}. On French complex
  predicates see \citew[\page 426--427]{GS2021a}.
}

I changed the section about Sign-Based Construction Grammar (SBCG) again. An argument about nonlocal
dependencies and locality was not correct, since \citet[\page 166]{Sag2012a} does not share all
information between filler and extraction side. The argument is now revised and presented as
Section~\ref{sec-local-feature-sbcg}. Reviewing \citew{MuellerCxG}, Bob Borsley pointed out to me that the \xargf is a way to
circumvent locality restrictions that is actually used in SBCG. I added a footnote to the section on
locality in SBCG.

A brief discussion of \citegen{Welke2019a-u} analysis of the German clause structure was added to the
chapter about Construction Grammar (see Section~\ref{sec-verb-position-cxg}).

The analysis of a verb-second sentence in LFG is now part of the LFG chapter
(Figure~\ref{Abbildung-V2-LFG} on page~\pageref{Abbildung-V2-LFG}) and not just an
exercise in the appendix. A new exercise was designed instead of the old one and the old one was
integrated into the main text.

I added a brief discussion of \citegen{Osborne2019a} claim that Dependency Grammars are simpler than
phrase structure grammars (p.\,\pageref{page-simplicity-dg}).

Geoffrey Pullum pointed out at the HPSG conference in 2019 that the label \emph{constraint"=based}
may not be the best for the theories that are usually referred to with it. Changing the term in
this work would require to change the title of the book. The label \emph{model theoretic} may be
more appropriate but some implementational work in HPSG and LFG not considering models may find the
term inappropriate. I hence decided to stick to the established term.

I followed the advice by Lisbeth Augustinus and added a preface to Part~II of the book that gives
the reader some orientation as to what to expect.

I thank Mikhail Knyazev for pointing out to me that the treatment of V to I to C movement in the
German literature differs from the lowering that is assumed for English and that some further
references are needed in the chapter on Government \& Binding. 

Working on the Chinese translation of this book, Wang Lulu pointed out some
typos and a wrong example sentence in Chinese. Thanks for these comments! 

I thank Bob Borsley, Gisbert Fanselow, Hubert Haider and Pavel Logacev for discussion and Ina Baier for a mistake
in a CG proof and Jonas Benn for pointing out some typos to me. Thanks to Tabea Reiner for a comment
on gradedness. Thanks also to Antonio Machicao y Priemer for yet another set of comments on the
second edition and to Elizabeth Pankratz for proofreading parts of what I changed.

~\medskip

\noindent
Berlin, 15th August 2019\hfill Stefan Müller



%      <!-- Local IspellDict: en_US-w_accents -->

%% -*- coding:utf-8 -*-

\section*{Foreword of the fourth edition}



% reference to Sag2020

% We thank Shalom Lappin and Richard Sproat for discussion of implementation issues.

% added footnote to gb chapter regarding the assignment of semantic role accross phrase boundary

% Thanks Andreas Pankau

% trincated English -> truncated English 19.01.2020

% fixed URLs, added DOIs

% e or t for trace -> t for trace

% be- und ent-laden sind keine Partikel sondern Präfixe
% Namen ersetzt und Frauen durch Eichhörnchen

% added references for DP approach (Brame, Hudson)

% Due to the work on Chinese, some index entries were fixed (Phenomenon/phenomenon)

I fixed several typos, added and updated URLs and DOIs in the book and in the list of references.
I added a footnote to Chapter~\ref{chap-GB} concerning the assignment of semantic roles
across phrase boundaries (footnote~\ref{fn-semantic-role-phrase-boundary} on
p.\,\pageref{fn-semantic-role-phrase-boundary}). I thank Andreas Pankau for discussion on this point.

I added a paragraph discussing John Torr's implementational work (pages~\pageref{page-torr-implementation-beginning}--\pageref{page-torr-implementation-end}). I thank Shalom Lappin and Richard Sproat for discussion of implementation issues.

A small paragraph for further reading was added to Chapter~\ref{chap-phrasal} on phrasal vs.\ lexical analyses.

Language Science Press will publish a handbook on Head-Driven Phrase Structure Grammar hopefully
later this year \citep*{HPSGHandbook}. It contains several chapters comparing other syntactic
theories to HPSG. I added the respective references to the further readings sections of the chapters
for Lexical Functional Grammar, Categorial Grammar, Construction Grammar, and Minimalism.

This edition is the first edition that uses precompiled trees. Setting this up was not
straightforward. I am really grateful to Sašo Živanović for helping me and adapting the
\texttt{forest} package so that everything runs smoothly and efficiently. This saves me a lot of
time and reduces the energy consumption of my computer dramatically.


~\medskip

\noindent
Berlin, 2nd September 2020\hfill Stefan Müller



%      <!-- Local IspellDict: en_US-w_accents -->

%% -*- coding:utf-8 -*-

\section*{Foreword of the fifth edition}

I want to thank Philip Kime for help with biber, the tool that Language Science Press is using for
creating lists of referecnes and for manipulating bibliography databases.

% added Ajdukiewicz35a-u to DP authors.

% changed some examples in 1-begriffe and other files

% 23.02.2021 changed AdvP -> [very] to AdvP -> [very,roof]

% 23.02.2021 fn regarding triangles in trees

% 26.01.2021 added Lötscher85 to dg chapter.

I fixed a mistake at the beginning of Section~\ref{sec-typeraising}: it now reads backward
application instead of forward application.

I fixed the Case Principle in the chapter on HPSG. The first two clauses did not mention that they
only apply to verbal heads.

I fixed some brackets in the Categorial Grammar derivation in Figure~\ref{Abbildung-CG-isst-der-junge-den-kuchen-jacobs}. There were
just too many brackets to keep track of everything \ldots. Thanks to Matthew Korte for spotting this!
Léonie Cujé found superflous brackets in Figure~\ref{abb-CG-Adjunktion}. They were removed. Thanks!


~\medskip

\noindent
Berlin, \today\hfill Stefan Müller



%      <!-- Local IspellDict: en_US-w_accents -->


%      <!-- Local IspellDict: en_US-w_accents -->
