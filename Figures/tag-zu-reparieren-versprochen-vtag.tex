%% -*- coding:utf-8 -*-
\documentclass[ number=1
                ,series=tbls,
	        %,blackandwhite
	        ,smallfont
                ,bibtex
	        %,draftmode  
		  ]{../Langsci/langscibook}                          


\usepackage{forest,lsp-forest-setup-texlive-2013,makros.2e,../Langsci/styles/langsci-optional}

\usetikzlibrary{tikzmark}

% compile with texlive 2013

% draw a grid for getting the coordinates
\usepackage{tikz-grid}


\begin{document}
\thispagestyle{empty}

\oneline{%
\menge{%
\forestset{begin draw/.code={\begin{tikzpicture}[baseline=(current bounding box.center)]}}
\hspace{1em}
\begin{forest}
[VP
	[NP$\downarrow$]
	[\subnode{vp1b}{VP}]]
\end{forest}
\hspace{1em}
\begin{forest}
[VP
	[NP$\downarrow$]
	[\subnode{vp2b}{VP}]]
\end{forest}
\hspace{1em}
\begin{forest}
[VP
	[NP$\downarrow$]
	[\subnode{vp3b}{VP}]]
\end{forest}
\hspace{1em}
\begin{forest}
[VP
	[NP$\downarrow$]
	[\subnode{vp4b}{VP}]]
\end{forest}
\hspace{1em}
\begin{forest}
[VP
	[\subnode{vprep}{VP}
		[$\epsilon$]
		[zu reparieren]]
	[\subnode{vpversprochen}{VP}
		[$\epsilon$]
		[versprochen]]]
\draw (-10.4,-3) to[grid with coordinates] (3,0.2);
\end{forest}
\begin{tikzpicture}[overlay,remember picture,out=-90,in=90,dashed]
\draw (vp1b) to (vpversprochen);
\draw (vp2b) to (vpversprochen);
\draw (vp3b) to (vprep);
\draw (vp4b) to (vprep);
\end{tikzpicture}
}}


\end{document}
