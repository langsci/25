%% -*- coding:utf-8 -*-
\title{Grammatical theory}

\subtitle{From transformational grammar to constraint-based approaches \newlineCover
\textit{\LARGE {\spaceskip=3.5pt Fifth revised edition} } }
\renewcommand{\lsImpressumCitationText}{%
Stefan Müller. \lsYear. \textit{Grammatical theory: From transformational grammar to
constraint-based approaches. Fifth revised and extended edition}. (Textbooks in Language Sciences~1). Berlin: Language Science Press.
}
\author{Stefan Müller}
\typesetter{Stefan Müller}
\translator{Andrew Murphy, Stefan Müller}
\proofreader{%
Viola Auermann, 
Armin Buch, 
Andreea Calude, 
Rong Chen,
Matthew Czuba,
Leonel de Alencar, 
Christian Döhler,
Joseph T. Farquharson,
Andreas Hölzl, 
Gianina Iordăchioaia, 
Paul Kay, 
Anne Kilgus, 
Sandra Kübler,
Timm Lichte, 
Antonio Machicao y Priemer,
Michelle Natolo,
Stephanie Natolo,
Sebastian Nordhoff,
Elizabeth Pankratz,
Parviz Parsafar, 
Conor Pyle,
Daniela Schröder,
Eva Schultze-Berndt,
Alec Shaw,
Benedikt Singpiel, 
Anelia Stefanova,
Neal Whitman,
Viola Wiegand}

\openreviewer{%
Armin Buch, 
Leonel de Alencar,
Andreas Hölzl,
Dick Hudson, 
Gianina Iordăchioaia, 
Paul Kay, 
Timm Lichte,
Antonio Machicao y Priemer, 
Andrew McIntyre, 
Arne Nymos,
Sebastian Nordhoff,
Neal Whitman
}

\BackTitle{Grammatical theory}
\BackBody{This book introduces formal grammar theories that play a role in current linguistic theorizing (Phrase Structure Grammar,
  Transformational Grammar/Government \& Binding, Generalized Phrase Structure Grammar, Lexical
  Functional Grammar, Categorial Grammar, Head-​Driven Phrase Structure Grammar, Construction
  Grammar, Tree Adjoining Grammar). The key assumptions are explained and it is shown how the
  respective theory treats arguments and adjuncts, the active/passive alternation, local
  reorderings, verb placement, and fronting of constituents over long distances. The analyses are
  explained with German as the object language.

The second part of the book compares these approaches with respect to their predictions regarding language
acquisition and psycholinguistic plausibility. The nativism hypothesis, which assumes that humans
posses genetically determined innate language-specific knowledge, is critically examined and
alternative models of language acquisition are discussed. The second part then addresses controversial issues
of current theory building such as the question of flat or binary branching structures being more appropriate, the question whether constructions should be treated on the phrasal or the lexical level, and the question whether abstract, non-visible entities should play a role in syntactic analyses. It is shown that the analyses suggested
in the respective frameworks are often translatable into each other. The book closes with a chapter
showing how properties common to all languages or to certain classes of languages can be
captured.

\bigskip
%~
%\smallskip
\vfill

%\noindent
%``With this critical yet fair reflection on various grammatical theories, Müller fills what has been a major gap in the literature.'' \href{http://dx.doi.org/10.1515/zrs-2012-0040}{Karen Lehmann, \textit{Zeitschrift für Rezensionen zur germanistischen Sprachwissenschaft}, 2012}


\medskip

\noindent
``Stefan Müller's recent introductory textbook, ``Grammatiktheorie'', is an astonishingly
comprehensive and insightful survey of the present state of syntactic
theory for beginning students.'' \href{http://dx.doi.org/10.1515/zfs-2012-0010}{Wolfgang Sternefeld und Frank Richter, \textit{Zeitschrift für Sprachwissenschaft}, 2012}


\medskip

\noindent
``This is the kind of work that has been sought after for a while. [\dots] The impartial and objective discussion offered by the author is particularly refreshing.'' \href{http://dx.doi.org/10.1515/germ-2011-537}{Werner Abraham, \textit{Germanistik}, 2012}

\medskip
\noindent
``These two volumes represent one of the first attempts since Sells' (1985) seminal work to provide
theoretical linguists and those who work closely with them with an overview of the general
representational machinery of contemporary frameworks and the key issues that separate those who
prefer one over another. In general, the presentation of empirical data and theoretical concepts is
highly accessible to scholar and student alike. The best use of these materials is for those seeking
to gain a better understanding of the core concepts that motivate the general representations
present in these frameworks. Although there are traits that are shared across many of these covered
here, there are also fundamental differences that persist. These volumes at the very least enable
those with different perspectives on key issues to engage in discussions and perhaps gain a better
understanding and appreciation of each other's research moving forward. In closing, contra
Sternefeld and Richter's (2012) somewhat pessimistic statements directed at an earlier version of
this work and toward the state of generative grammar a priori, I view Müller's work here in a
positive light as a conduit that has the potential to bring formal linguists together to gain a
fuller appreciation of theoretical work beyond their own immediate communities.'' \href{http://doi.org/10.5334/gjgl.414}{Michael T. Putnam, \emph{Glossa}, 2017}


\medskip
\noindent
``Here is a grand piece of work that should be read and taken seriously by grammarians of all
stripes. It is a highly welcome antidote to the dominant trend of splendid isolation among the
various schools of grammar theory in the past decades. This up-to-date, analytic, comparative and
critical review of current major schools of grammar research is an offer one can't refuse.'' Hubert
Haider, 2020


}
\dedication{For Max}
\renewcommand{\lsISBNdigital}{978-3-96110-402-4}
%\renewcommand{\lsISBNhardcoverOne}{978-3-96110-075-0}
%\renewcommand{\lsISBNhardcoverTwo}{978-3-96110-076-7}
% \renewcommand{\lsISBNsoftcoverOne}{978-3-96110-203-7} % 3rd edition
% \renewcommand{\lsISBNsoftcoverTwo}{978-3-96110-204-4} % 3rd edition
%\renewcommand{\lsISBNhardcover}{978-3-96110-277-8} % 4th ed., one-volume ISBN 	
\renewcommand{\lsISBNsoftcover}{978-3-98554-060-0} % 5th ed., one-volume ISBN
\renewcommand{\lsBookDOI}{10.5281/zenodo.7376662} % 5th edition
\renewcommand{\lsSeries}{tbls} % use lowercase acronym, e.g. sidl, eotms, tgdi
\renewcommand{\lsSeriesNumber}{1} %will be assigned when the book enters the proofreading stage
\renewcommand{\lsID}{380} % github, paperhive, URL
%\renewcommand{\paperhivetext}{paperhive.org/documents/remote?type=langsci&id=380} % contact the coordinator for the right number
%\renewcommand{\githubtext}{www.github.com/langsci/25} % contact the coordinator for the right
%number
\renewcommand{\paperhivetext}{Errata:~\href{https://paperhive.org/documents/remote?type=langsci&id=25}{paperhive.org/documents/remote?type=langsci\&id=25}}
\renewcommand{\githubtext}{Source code available from \href{https://www.github.com/langsci/25}{www.github.com/langsci/25}}
\renewcommand{\lsSeries}{tbls}
%\renewcommand{\lsYear}{2023}


%\vspace{-8cm}\textit{Second revised and augmented edition}

%      <!-- Local IspellDict: en_US-w_accents -->
